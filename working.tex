\documentclass{article}

\usepackage{hyperref}

\title{Working Paper on Practically Modelling Adaptations}

\author{Tim Storer, Tom Wallis}


\begin{document}

\section{Introduction}

\section{Related Literature}

\subsection{Obashi}
Obashi\cite{ObashiMethodology} is a dataflow modelling methodology and accompanying tool with a focus on dataflow in
business \& IT. They happen to be our sponsors.\par

Obashi makes the claim that IT exists solely for the purpose of enabling dataflow between business assets. Whether we
agree with that in all cases, it's a good introduction to the philosophy their framework is based on. Obashi models
dataflows as a graph, with six types of relations between its elements. Related nodes imply things like functional
dependancies, dataflow, logical/physical groupings and so on.\par

One of the focuses of Obashi is a simplicity in presentation; some parts of their methodology exist purely for the
simplification of a display of a system, such as their ``element persistence''. When elements don't exist on a given
graphical representation, that doesn't imply that they're not still an important part of, or potentially affecting, the
system in other areas. Using rules like these, Obashi can selectively display subgraphs with contextual meaning, which
can loosely be thought of as one of Applied Systems Theory's ``aspects''\cite{dekkers:appliedsystemstheory}.

Obashi see dataflow the way some people might see the flow of a commodity such as oil, and like a model of that kind of
flow, Obashi's useful for things like modelling security, profit, and system improvements. It's also powerful as a way
to capture and observe system information, without then performing analysis on the captured information. Importantly, an
Obashi model comprises a metric space where information can be recorded on relationships and elements and then used to
aggregate and calculate values across a network in the model. Combined with the advantage of actually being capable of
producing a model via tooling that actually exists (unlike other modelling concepts), Obashi represents modelling that's
limited in scope (focusing on Business \& IT), but succeeding in pragmatism.

\subsection{Modelling with Bigraphs}

Bigraphs are a modelling technique devised by Robin Milner around 2001. They're a mix of clever category theory for the
\emph{composition} of systems and graphical / algebraic expressions for the actual construction and engineering of
systems. A good introduction to them is \cite{milner:bigraphsandtheiralgebra}; the canonical reference is
\cite{milner:spaceandmotion}. Interestingly, while they're sometimes used for system modelling, they're originally
designed for the study of process calculi.\par

Bigraphs' graphical representation is the easiest to understand; systems which are \emph{spatially} nested are
represented as nodes of a graph held inside other nodes. Links between these nodes --- usually used to represent a
shared resource --- are hyperlinks which multiple nodes can share. In this way, the graphical representation of a
bigraph can show complicated systems with intricate dependancies. The way a system changes, referred to as its
\emph{motion}, is represented outside of the graphical representation of a bigraph itself: a bigraph is paired with a
library of ``reaction rules'' which show how any subsystem of a bigraph can change in a given tick.\par

(Interestingly, there's an implicit separation of concerns here between a system's structure and its behaviour, which is
nice\ldots{})\par

This representation is backed up by a sophisticated category theoretic foundation; bigraphs are unusual in that they've
got a nice graphical format but they're actually well founded mathematically. The maths here lets us make compositions
in well-defined ways. This is actually important mostly for applying the reaction rules. It lets us define subsystems of
a large system and swap it out for another subsystem in a way that won't break the model (although encoding high-level
meaning into a bigraphical model sits somewhere on the spectrum of hard and impossible, so the guarantee is really that
applying the reaction rule is ``safe''; it won't break the ``code'' of the bigraph, and if your reaction rules are sane
then there should be nothing to worry about. In theory.)\par

Finally, the definitions from the category theory, and the structure implied by the graphical notation, gives a nice
algebraic format that we can represent. The fact that bigraphs have an algebra is really important, because there's
scope here to define languages on the algebra like regular languages are defined on the lambda calculus. As an aside,
bigraphs can also represent calculi like the \(\pi\)-calculus, ambient calculi, and some more systems-y things like
petri nets. There's potentially scope for some research around developing a high-level bigraph language straight from
the algrabra, rather than something like BigraphER (Michele's language\cite{michele:bigrapher})

There are other properties of the models --- signatures to identify different ``types'' of node (making a weak type
system in effect), ports that allow nodes to communicate, and place / link graphs which combine to actually make the
bigraph --- are best picked up from \cite{milner:bigraphsandtheiralgebra} in my opinion. It's turtles all the way down
with bigraphs\ldots{}

\subsubsection{Some Bi-gripes with Bi-graphs}

% Bigraphs' temporal model is weak!
Bigraphs have an implicit temporal model, which is that a reaction rule effectively takes one ``tick'' to complete. This
feels really incomplete; for a proper systems modelling framework, and one that's pragmatic outside of academic
interests, we'd need something that can contend with the demands of things like business models where a more robust
temporal model is crucial.

% They can be clumsy to actually use
Bigraphs also feel like they'd be really clumsy to use. A part of that is that they've got this rigorous mathematical
model that you don't want to break, so just saying ``I'll just add this into the model'' like we might with a code-based
model --- where we actually control the modelling paradigm --- isn't so straightforward. An example is creating what
I've been referring to as ``metric bigraphs'': bigraphs don't make a metric space, so you can't measure things across
the hyperlinks and calculate things like times, costs, and distances across nodes. Michele's done some work on this (and
perplexingly he doesn't think there's much use for it and never published --- I wonder what he's thinking, and whether
he knows something I don't\ldots{}) and the fact that \emph{there was work to do on it} says a lot about the fact that,
to make bigraphs \emph{useful}, you seem to need to extend them.\par

We hit an interesting problem as a result, which is that there's an extension of bigraphs for sharing
(\cite{michele:sharingbigraphs}), directed bigraphs (\cite{grohmann:directedbigraphs}), and presumably others I haven't
come across, but there's no guarantee that they're mutually compatible because the underlying formalism to make sure
everything's tickety-boo hasn't been done! Any extension to bigraphs as a modelling paradigm is either surface-level,
written in a high level and boiling down to something hacked together with the original formalism --- Michele has a
turning-reaction-rules-on-and-off thing that works this way in BigraphER with a json API --- or takes a lot of work to
build on top of the original formalism, and then you lose the benefits of anybody else's work.\par

In short: \emph{They're clumsy.}\par

% Unclear how you get very detailled statistics in an easy way about a % simulation
Also, a detail of that --- that bigraphs don't make a metric space --- means that collecting meaningful / useful
statistics about a model is a little trickier than it would be with something like PyDySoFu.\par

Example: what's the implication of a given system change, on average, with regards the latency people using router A
experience? ``Latency'' isn't really feasible to easily define in the model, and making \emph{calculations} with a value
like that --- and making future ``decisions'' in the model conditional on that value --- isn't really feasible. It might
be that we use something like BigraphER's clever json API to turn reaction rules on or off in a higher-level language
that interfaces with the bigraph, and then constantly rewrite and change some value as the system adapts, turning
reaction rules on and off to guide the adaptation, but then we've written an entire wrapper around the bigraph model to
get our stats anyway. It's way better to get the statistics as a part of the underlying model.\par

(Sidenote: maybe there's room, then, to write a high-level modelling paradigm that captures information like that and
gives us the interface we'd need to a tool like BigraphER, so that more complicated models can be written and we hack
together the low level stuff as we need to. Michele claims that bigraphs are very fast, ast least his OCaml
implementation, so that could be feasible. I wonder whether that's a PhD-worthy research project in-and-of
itself\ldots{}?)\par

% Conclusion: useful, but like Haskell.
Ultimately: it's useful like Haskell's useful. You can do some cool things in it and it's got some lovely guarantees if
you care about that kind of thing, but doing anything large in it would be a nightmare because you'd constantly be
fighting the formal model that underpins it all. Ever tried Haskell I/O?\par

\subsection{BPMN}

\subsection{Petri Nets and Process Calculi}

\subsection{Modelling philosophies}

\subsubsection{General Systems Theory}
\subsubsection{Complex Adaptive Systems}
\subsubsection{Applied Systems Theory}
\subsubsection{Agent-based Models}
\subsubsection{Holon-based Models}


\section{Anatomy of a Model}
% What do these models have in *common*?

\subsection{Temporal Models}
\subsection{Space \& Relationships}
\subsection{Motion, Behaviour \& Adaptation}
\subsection{Representations of Models} % Algebras, graphical representations, maths, code, proprietary tooling, paper...
                                       % pros and cons of them all.


\section{Memo of Understanding} % Any ideas we've had or definitions we've % agreed on and want to record.

\subsection{Terminology}

Some terms:\par
\begin{description}
\item[Capturing]\hfill\\ Collecting information about a system and constructing a representative model using it. (Term
  basically taken from \href{https://gettingthingsdone.com/five-steps/}{the GTD definition})
\item[Analysing]\hfill\\ Learning something about the captured system by any kind of assessment the modelling
  environment permits
\end{description}

\subsection{Our topic}

\ldots{}something around adaptation? Modelling adaptation is hard.

\bibliography{lib}
\bibliographystyle{plain}
\end{document}
\documentclass{article}

\title{Working Paper on Practically Modelling Adaptations}
\author{Tim Storer, Tom Wallis}

\begin{document}

\section{Introduction}

\section{Related Literature}

\subsection{Modelling with Bigraphs}

Bigraphs are a modelling technique devised by Robin Milner around 2001. They're
a mix of clever category theory for the \emph{composition} of systems and
graphical/algebraic expressions for the actual construction and engineering of
systems. A good introduction to them is \cite{milner:bigraphsandtheiralgebra};
the canonical reference is \cite{milner:spaceandmotion}.

Bigraphs' graphical representation is the easiest to understand; systems which
are \emph{spatially} nested are represented as nodes of a graph held inside
other nodes. Links between these nodes --- usually used to represent a shared
resource --- are hyperlinks which multiple nodes can share. In this way, the
graphical representation of a bigraph can show complicated systems with
intricate dependancies. The way a system changes, referred to as its
\emph{motion}, is represented outside of the graphical representation of a
bigraph itself: a bigraph is paired with a library of ``reaction rules'' which
show how any subsystem of a bigraph can change in a given tick.

(Interestingly, there's an implicit separation of concerns here between a
system's structure and its behaviour, which is nice\ldots{})

\subsubsection{Some Bi-gripes with Bi-graphs}

% Bigraphs' temporal model is weak!

% They can be clumsy to actually use

% Unclear how you get very detailled statistics in an easy way about a
% simulation 



% \subsection{Adaptations in Formal Methods}
% \subsection{Adaptations in CAS}
% \subsection{Adaptations in Programs}
% \subsection{Adaptations in Business Models}

\section{Memo of Understanding} % Any ideas we've had or definitions we've
                                % agreed on and want to record.

\bibliography{lib.bib}
\end{document}